\documentclass{article}
\usepackage{xcolor}
\usepackage{tikz}
\usepackage{parskip}
\usepackage{amsmath}
\usepackage{amsthm}
\usepackage{amssymb}
\usepackage{mathtools}
\usepackage{fancyhdr}
\usepackage[%paperheight = 59.4cm,
            %paperwidth = 42cm,
            %includehead,
            nomarginpar,
            textwidth=15cm,
            headheight=10mm]{geometry}


\begin{document}

\pagestyle{fancy}
%\fancyhead{}\fancyfoot{}

\fancyhf[OHC]{Christopher Munoz Optimizations - General notes}
\setcounter{section}{1}
\setcounter{subsection}{1}
\setcounter{subsubsection}{0}
\subsubsection{}
\textbf{On Feasibility Sets and Points} - Given an objective function with constraints, the feasible region or feasible set are the set of all points that satisfiy all constraints. The feasible set is denoted by $S$ in our text.

A feasible point $\bar{x}$ in an inequality constraint $g_i(\bar{x})$ is said to be binding or active if $g_i(\bar{x}) = 0$, this is also called being on the boundary of the constraint. 

Likewise a feasible point is said to be unbinding or inactive if $g_i(\bar{x}) > 0$ or $g_i(\bar{x}) < 0$, this is called being interior of the constraint.

\textbf{On optimization problems in general} - There really isn't any difference between minimization and maximization problems the problem 




\begin{align}
    \max_{x \in S}
\end{align}
\end{document}
