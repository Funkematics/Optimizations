\documentclass{article}
\usepackage{xcolor}
\usepackage{graphicx}
\usepackage{float}
\usepackage{tikz}
\usepackage{parskip}
\usepackage{amsmath}
\usepackage{amsthm}
\usepackage{amssymb}
\usepackage{mathtools}
\usepackage{fancyhdr}
\usepackage[%paperheight = 59.4cm,
            %paperwidth = 42cm,
            %includehead,
            nomarginpar,
            textwidth=15cm,
            headheight=10mm]{geometry}


\begin{document}
 
\pagestyle{fancy}
%\fancyhead{}\fancyfoot{}

\fancyhf[OHC]{Christopher Munoz WRH4 Optimization}
\textbf{Problem 2.2:} \\
Theorem: Let $Z$ be an $n$ x $r$ null-space matrix for the matrix $A$. If $Y$ is any invertible $r$ x $r$ matrix,  $\hat{Z} = ZY$ is also a null-space matrix for A.

Proof: Given that Z is an $n$ x $r$ null-space matrix for matrix $A$, we know that $AZ = 0$, we also know that $Y$ is an invertible matrix. We need to show that $A\hat{Z} = 0$ or $AZY = 0$ for $\hat{Z} = ZY$ to be a null-space matrix for A. Consider the following:
\begin{align*}
    A \hat{Z} = A(ZY) \\
    A(ZY) = (AZ)Y \\
    A(ZY) = 0Y \\
    AZY = 0 \text{ or } A\hat{Z} = 0
\end{align*}
Thus proving $\hat{Z} = ZY$ is also a null-space matrix for A. \newline

\textbf{Problem 3.1:} We will compute a basis for the null space for the following matrices(Denoted with A) using variable reduction:

\textbf{\{i\}}
\begin{align*} A =
    \begin{bmatrix}
        1 & 1 & 1 & 1 \\
        1 & -1 & -1 & 1 \\
        0 & 1 & 0 & 1
    \end{bmatrix} \Rightarrow
    \begin{bmatrix}
        1 & 1 & 1 & 1 \\
        0 & -2 & -2 & 0 \\
        0 & 1 & 0 & 1
    \end{bmatrix} && R_1 - R_2 \xrightarrow{} R_2 \\
    \begin{bmatrix}
        1 & 1 & 1 & 1 \\
        0 & -2 & -2 & 0 \\
        0 & 0 & -1 & 1
    \end{bmatrix} && \frac{R_2}{2} + R_3 \xrightarrow{} R_3 
\end{align*}\begin{align*}
    x_1 + x_2 + x_3 + x_4 = 0 && -2x_2 - 2x_3 = 0 && -x_3 + x_4 = 0 && x_4 = x_4 \\
    x_4 = t && x_3 = t && x_2 = -t && x_1 = -t
\end{align*}
Thus null(A) $ = t\begin{bmatrix} -1 \\ -1 \\ 1 \\ 1\end{bmatrix}$ for $t \in \mathbb{R}$ \newline

\textbf{\{ii\}}
\begin{align*} A = 
    \begin{bmatrix}
        1 & 1 & 1 & 1
    \end{bmatrix}
\end{align*}
\begin{align*}
    x_1 + x_2 + x_3 + x_4 = 0 &&
    x_2 = x_2 && x_3 = x_3 && x_4 = x_4 \\
    x_2 = s && x_3 = t && x_4 = u && x_1 = -s -t - u
\end{align*}
Thus null(A) $ = s\begin{bmatrix} -1 \\ 1 \\ 0 \\ 0\end{bmatrix} + t\begin{bmatrix} -1 \\ 0 \\ 1 \\ 0\end{bmatrix} +u\begin{bmatrix} -1 \\ 0 \\ 0 \\ 1\end{bmatrix} $   for $s,t,u \in \mathbb{R}$ \newline
\textbf{\{iii\}}
\begin{align*} A = 
    \begin{bmatrix}
        1 & 1 & 1 & 1 \\
        1 & -1 & -1 & 1
    \end{bmatrix} \Rightarrow
    \begin{bmatrix}
        1 & 1 & 1 & 1 \\
        0 & 2 & 2 & 0
    \end{bmatrix} && R_1 - R_2 \xrightarrow{} R_2
\end{align*}
\begin{align*}
    x_1+x_2+x_3+x_4 = 0 && 2x_2 + 2x_3 = 0 && x_3 = s && x_4 = t \\
    x_2 = -x_3 = -s && x_1 - s + s + t = 0 && x_1 = s - s - t && x_1 = -t
\end{align*}
Thus null(A) $ = s\begin{bmatrix} 0 \\ -1 \\ 1 \\ 0\end{bmatrix} + t\begin{bmatrix} -1 \\ 0 \\ 0 \\ 1\end{bmatrix} $   for $s,t \in \mathbb{R}$ \newline

\textbf{\{iv\}}
\begin{align*} A =
    \begin{bmatrix}
        1 & 1 & 1 & 1 \\
        2 & 0 & 0 & 2 \\
        1 & -1 & -1 & 1
    \end{bmatrix} \Rightarrow
    \begin{bmatrix}
        1 & 1 & 1 & 1 \\
        2 & 0 & 0 & 2 \\
        0 & -1 & -1 & 0
    \end{bmatrix} && \frac{R_2}{2} - R_3 \xrightarrow{} R_3 \\
    \begin{bmatrix}
        1 & 1 & 1 & 1 \\
        0 & 2 & 2 & 0 \\
        0 & -1 & -1 & 0
    \end{bmatrix} && 2R_1 - R_2 \xrightarrow{} R_2 \\
    \begin{bmatrix}
        1 & 1 & 1 & 1 \\
        0 & 2 & 2 & 0 \\
        0 & 0 & 0 & 0
    \end{bmatrix} 
    && \frac{R_2}{2} + R_3 \xrightarrow{} R_3 \\
    \begin{bmatrix}
        1 & 1 & 1 & 1 \\
        0 & 1 & 1 & 0 \\
        0 & 0 & 0 & 0
    \end{bmatrix} 
    && \frac{R_2}{2} \xrightarrow{} R_2 \\
    \begin{bmatrix}
        1 & 0 & 0 & 1 \\
        0 & 1 & 1 & 0 \\
        0 & 0 & 0 & 0
    \end{bmatrix}
    && R_2 - R_1 \xrightarrow{} R_1
\end{align*}
\begin{align*}
    x_1 + x_4 = 0 && x_2 + x_3 = 0 && x_3 = s && x_4 = t \\
    x_2 = -s && x_1 = -t
\end{align*} Thus null(A) $ = s\begin{bmatrix} 0 \\ -1 \\ 1 \\ 0\end{bmatrix} + t\begin{bmatrix} -1 \\ 0 \\ 0 \\ 1\end{bmatrix} $   for $s,t \in \mathbb{R}$ \newline



\end{document}
